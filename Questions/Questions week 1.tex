\documentclass{article}
\usepackage[utf8]{inputenc}
\usepackage{fullpage}

\title{Cognitive modeling: intelligent questions week 2}
\author{Steven Bosch (s1861948)}

\begin{document}
	
\maketitle
\section{Time Perception: Beyond Simple Interval Estimation}
In the description of the interval estimation experiments it is mentioned that the participants were forbidden from counting. In general telling people not to think of something immediately makes them think of that particular thing. Would telling them not to count not have a similar effect? If not, would it have any other effect on the outcome of the experiment results, making it more plausible that the model resembles the experiment results, since the model uses some form of counting as well? In other words: would the model still correctly approximate the test results if the participants would not have counted because they did not know they had to keep track of time instead of just being told not to count?

\section{Traces of times past: Representations of temporal intervals in memory}
While the aim of the paper is to investigate how representations of time intervals are learned and represented (the results indicate the use of pools of recent experiences instead of single memory traces), my question concerns the presupposed assumption concerning memory as a basis for this cognitive task in general.
It is explained that in this research the memory component for models of time perception are modelled by general models of memory. My question is as follows: Are there any recorded cases of patients with brain damage to areas associated with short-term memory who also showed difficulty in time estimation tasks? This would be interesting, since such patients would back up the assumption that representing time intervals involves `standard' memory processes.

\end{document}