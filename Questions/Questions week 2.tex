\documentclass{article}
\usepackage[utf8]{inputenc}
\usepackage{float}
\usepackage{graphicx}
\usepackage{fullpage}
\usepackage{caption}

\title{Cognitive modeling: intelligent questions week 3}
\author{Steven Bosch (s1861948)}

\begin{document}
	
\maketitle
\section{Shi et al. \textit{Bayesian optimization of time perception}}
The paper suggests that the likely reason for contextual calibration is an effort to improve timed performance by reducing the overall error using a Bayesian approach. Could other cognitive processes, besides time perception, in which this contextual calibration might occur also be explained using a Bayesian approach? Things like colour and brightness of light or tone and intensity of sound?

\end{document}