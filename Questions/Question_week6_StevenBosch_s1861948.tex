\documentclass{article}

\title{Question week 6}
\author{Steven Bosch (s1861948)}

\begin{document}
\maketitle
\noindent My question concerns the paper by Buhusi \& Meck (2005):\\

\noindent The coincidence-detection model assumes the cortical oscillators to be synchronized at the onset of a trial and to oscillate at a fixed frequency throughout the criterion interval. It also assumes experience-dependent changes in cortico-striatal transmissions to make the striatal neurons be more likely to detect the specific pattern of activation of cortical oscillators at the time of reward delivery and/or feedback. (Page 8)

My question concerns the manner in which the uniqueness of the signal is established for the cortico-striatal transmissions to induce the detection by the striatal neurons. In the next paragraph it is mentioned that the changes in cortico-striatal transmissions are ascribed to the activity of dopaminergic neurons showing a "characteristic pattern", but it is not mentioned whether this pattern is unique, and if it is, how it is established. Therefore my question is: are these patterns unique and if so, in what sense? (Frequency, amplitude, area of cortical oscillatons?) And if not, how then would the cortico-striatal transmissions `know' when to "make the striatal neurons more likely to detect the specific pattern of activation of cortical oscillators at the time of reward delivery", as is mentioned on page 8?

\end{document}